\documentclass[10pt]{article}
\usepackage{amsmath}
\usepackage{amssymb}
\usepackage[backend=bibtex, style=authoryear]{biblatex}

% \addbibresource{~/Dropbox/TAGreview/PhDreferences.bib}

\textwidth=470pt
\oddsidemargin=0pt
\topmargin=0pt
\headheight=0pt
\textheight=650pt
\headsep=0pt

\title{Statement of Research}


\author{Nicholas Santantonio}
\date{\today}


\newcommand{\gxe}{G$\times$E}
\newcommand{\gxg}{G$\times$G}


\begin{document}

\section*{\centering Statement of Teaching}
\begin{center} Nicholas Santantonio \end{center}
% \subsection*{Introduction}

\noindent Plant breeders have always been generalists, combining a range of plant sciences with genetics to identify farmers' needs and produce products to meet those needs. However, the range of skills required in the field is rapidly increasing. Competitive plant breeders are now expected to be proficient in statistics, programming, bioinformatics and computational biology along with the traditional skills in physiology, pathology, agronomy and genetics. Currently, most plant breeding students do not acquire in-depth statistics and programming skills until graduate school, impeding their progress while they learn to grapple with these new languages. As educators, we must actively update our curriculum to produce students who will be competitive in the workplace if Cornell is to continue to be a leader in the plant breeding community. This is especially true today, where plant breeding has seen a rapid and unprecedented paradigm shift in recent years away from tinkering with single genes and toward whole genome engineering.

% While single- and oligo-gene mendelian triats are abundant, why should students be limited to this type of genetic variation?  


\subsection*{Teaching Philosophy}

It is important for students to be exposed to plant breeding ideas from multiple perspectives, and they must given the opportunity to demonstrate critical thinking on many levels. While some students may show analytical thinking and synthesis on exams, others may shine in more hands-on projects. As a student, my most memorable courses were those that had a semester long project that was presented to the class at the end of the semester. These long term projects are invaluable for assessing student understanding of the course material. Regularly assigned homework and hands-on labs are also crucial for estimation of the pace and overall comprehension of the material presented so that adjustments can be made where necessary. 
% Hands-on labs are also useful for evaluating individual students, as it becomes easier to spot students that may need extra attention if they are struggling in labs as well as on homework assignments. 

The greenhouse and field aspects of plant breeding are difficult to present in lecture courses. One of the benefits of having greenhouse and field space on campus is that students can be brought out of the classroom to see what breeding looks like in practice, even though Ithaca weather mostly restricts this to fall semesters. The computational aspects of plant breeding can be incorporated into the curriculum through homework assignments, labs and semester long projects. Public databases provide invaluable resources where students can find data sets for use in projects that teach them how to work with real, and often large data sets. Computer simulations are another useful way to evaluate comprehension, where in order to simulate a system correctly, the student must understand that system well.

All courses I instruct will contain a term project along with exams and other in-class labs and homework assignments. For quantitatively oriented courses, this project would consist of a hands-on computational component, where students either find a dataset or use their own data to explore the ideas we cover in the course. They would then be asked to present their results in written and oral formats that mirror typical scientific communication. For courses without a significant quantitative aspect, term projects would be proposal oriented, where the students would be expected to construct a hypothesis, give background information on merit and propose a way to ask their question while also considering potential pitfalls. %These proposals would be guided throughout the semester to ensure students can 

% Any course related to quantitative genetics should have a computational aspect to it, 

\subsection*{Courses}

There are currently two courses offered by the plant breeding section that will need instructors in the near future, PLBRG2010 and PLBRG4030. I worked closely with Susan McCouch as a TA to teach PLBRG2010 in fall of 2017, and would be honored to continue as that course's instructor. My background would also be appropriate to teach the plant breeding staples of PLBRG4030 while incorporating a more hands-on computational focus to further prepare students for study of plant breeding on a quantitative level. 

% My expertise is quantitative in nature, and I would want to incorporate more of the quantitative element to the course so that students are exposed to what the plant breding field looks like today, and what it may look like in the future.

I propose to teach two courses, one of either PLBRG2010 or PLBRG4030, as well as a new cross-listed, ``Introduction to Quantitative Genetics in Plant Breeding'' 4000 level course. The new course would prepare senior undergraduates and first year graduate students for further study of quantitative genetics and would cover some background material currently needed for PLBRG7160. I would like to work with Assistant Professor Kelly Robbins, as well as faculty in Biometry and Computational biology to determine what material should be covered in this cross-listed course, therefore allowing PLBRG7160 to be an advanced course that provides more rigorous mathematical detail to quantitative genetics.

% \section*{(PLBRG2010)}
% \section*{PLBRG 2XXX - Genome to Table: Technological advances in plant breeding}
\subsection*{PLBRG 2010 - Plants, Genes, and Global Food Production}

If I were to instruct PLBRG2010, the course would continue much of the structure built by Professor Susan McCouch, while shifting the focus toward the second half of the semester to be slightly more quantitative in nature. The course would be taught from a historical perspective, allowing the natural progression of plant breeding over the past century to be a teaching tool for how we arrived at the methods we use today. The term project will be an exploration into a ``scientific, societal, or technological challenge'' related to plant breeding in the form a research proposal. Students will submit outlines and meet with the instructor mid-semester to evaluate the appropriateness of the proposed topic. ``Crop of the Day'' presentations will be assigned during the first week, where each student will choose a crop on which to present information about the origin, biology and breeding methods/goals of the crop, to be given at the beginning of each class. By the end of the semester, students will be able to think critically about how breeding technology changes through time have impacted the discipline and the societal view thereof. 

% Topics covered in the course would include a brief review of Mendelian genetics, domestication, genetic diversity, mating systems, the Green Revolution, biotechnology, intellectual property, gene mapping, genetics of complex traits, and introduction to new plant breeding technologies such as genomic selection and high throughput phenotyping.

%The course will introduce students to plant breeding as a discipline and cover the basics of domestication, genetic diversity, and mating systems before shifting to a quantitative perspective to introduce students to these topics such that they may seek further quantitative and computational instruction before they graduate. %Topics covered in the course would include a brief review of Mendelian genetics, domestication, genetic diversity, mating systems, the Green Revolution, biotechnology, intellectual property, gene mapping, genetics of complex traits, and introduction to new plant breeding technologies such as Genomic selection and high throughput phenotyping.

% \textbf{Objective:}  Students will be able to express how breeding technology changes through time have impacted the discipline and the societal view thereof. 

% Student performance will be evaluated by three exams, regularly assigned homework, a term project, and a single a ``Crop of the Day'' presentation. The term project will be an exploration into a scientific, societal, or technological challenge related to plant breeding a in the form a research proposal. Students will submit outlines and meet with the instructor mid-semester to evaluate the appropriateness of the proposed topic. ``Crop of the Day'' presentations will be assigned during the first week, where each student will choose a crop on which to present information about the origin, biology and breeding methods/goals of the crop, to be given at the beginning of each class.


\subsection*{PLBRG 4030 - Genetic Improvement of Crop Plants}


As for PLBRG4030, I would like to incorporate more hands-on computational labs and a term project on top of the literature review framework put into place by Professor Larry Smart. I appreciate the primary literature approach to the course, but may revise the series of papers used to introduce ideas. To augment their understanding of course material, a regular weekly lab will be implemented in which students use computational tools to analyze example datasets. The term project will consist of groups of 2-4 students finding a genotype-phenotype dataset, and working to develop a genotype to phenotype map and assess genomic predictability throughout the semester using available software. Students will be able demonstrate critical and analytical thinking of plant breeding methods and ideas, with the ability to synthesize when given new information or problems. 
% The course will introduce students to plant breeding as a discipline and cover the basics of domestication, genetic diversity, and mating systems before shifting to a quantitative perspective to introduce students to these topics such that they may seek further quantitative and computational instruction before they graduate. %

% \textbf{Objective:} Students will be able demonstrate critical thinking of plant breeding ideas, with the ability to synthesize when given new information or situations. Students will be able to express how breeding technology changes through time have impacted the discipline and the societal view thereof. 

% Student performance will be evaluated by three exams, regularly assigned homework, a term project, and a single a ``Crop of the Day'' presentation. The term project will be an exploration into a scientific, societal, or technological challenge related to plant breeding a in the form a research proposal. Students will submit outlines and meet with the instructor mid-semester to evaluate the appropriateness of the proposed topic. ``Crop of the Day'' presentations will be assigned during the first week, where each student will choose a crop on which to present information about the origin, biology and breeding methods/goals of the crop, to be given at the beginning of each class.


% The course will introduce students to the basics of plant breeding from a historical perspective and transition to the quantitative and computational aspects of the rapidly evolving field. 

% \begin{itemize}

% 	\item A brief review of Mendelian genetics (Mendel, Morgan, etc.)
% 	\begin{itemize}
% 		\item 1 and 2 gene inheritance
% 		\item chromosomes, linkage
% 	\end{itemize} 

% 	\item Domestication and genetic diversity (Nikolai Vavilov, Cary Fowler, etc.)
% 	\begin{itemize}
% 		\item Centers of origin 
% 		\item Domestication traits
% 		\item Gene banks: seeds vs clonal species
% 	\end{itemize} 

% 	\item Mating systems and ploidy (with pollination lab)
% 	\begin{itemize}
% 		\item Inbred crops
% 		\item Outcrossing crops
% 		\item Hybrids  
% 	\end{itemize} 

% 	\item Green Revolution (Norman Borlaug, etc.)
% 	\begin{itemize}
% 		\item Impact of inorganic nitrogen
% 		\item Asia
% 		\item Africa
% 	\end{itemize} 

% 	\item Biotechnology \& Intellectual Property
% 	\begin{itemize}
% 		\item Traditional GMOs (with a contrast to mutagenesis)
% 		\item Gene editing - GMO or not?
% 		\item PVP and Patents 
% 	\end{itemize}

% 	\item Locating genes (with linkage mapping lab)
% 	\begin{itemize}
% 		\item Linkage mapping 
% 		\item GWAS (intro)
% 	\end{itemize} 

% 	\item Complex traits (Fisher, etc.)
% 	\begin{itemize}
% 		\item Additivity
% 		\item Dominance
% 	\end{itemize} 

% 	\item Introduction to new(er) technologies
% 	\begin{itemize}
% 		\item Marker-assisted selection (MAS)
% 		\item Genomic selection (GS)
% 		\item High throughput phenotyping (HTP)
% 	\end{itemize} 

% \end{itemize}




\subsection*{PLBRG 4$\cdots$ - Introduction to Quantitative Genetics for Plant Breeding }

Finally, I want to work with Assistant Professor Kelly Robbins and faculty from biometry and computational biology to produce another entry level graduate or cross-listed course that will aim to prepare students for quantitative and computational genetics. Currently, there is a single graduate level course on quantitative genetics offered by the Plant Breeding and Genetics Section, under the instruction of Dr. Kelly Robbins. That course would benefit greatly from a introductory course, such that more time can be dedicated to an in-depth mathematical treatment of quantitative genetics theory for students who would like to specialize in quantitative genetics. This introductory course would provide the background theory, coding and application of known methods, while reserving focus on mathematical proofs and Bayesian methods for the advanced course. I envision essentially all plant breeding graduate students to take this course to gain the basic skills, while the advanced class would be available for those who wish to pursue the discipline further. Covered topics may include review of probability, single gene model, genetic variances and covariances, linear regression, linear mixed models, genome wide association and genomic prediction.
 % who do not wish to pursue the discipline further, but need the basic tools. The introductory QG course will focus on coding and application of known methods

The course would include a weekly computational lab and a paired student term project which will be developed throughout the semester. All homework assignments will contain both theoretical and computational problems, that must be submitted as typed documents in Markdown, \LaTeX\ or similar format. The term project will be an investigation into a quantitative genetics problem of their choosing, where the students will be free to use existing data, synthesize their own data and/or simulate a genetic system using tools they will be learning and writing along the way. Proposal outlines will be submitted half way through the semester to evaluate project appropriateness, and students will be encouraged seek input from the instructor throughout the semester. Late semester weekly labs will be used for working on projects, where code readability and reproducibility will be emphasized.

At the semester conclusion, students will demonstrate mastery of regression and linear mixed models to solve GWAS and GS problems. Students will be able to build design matrices and fit linear mixed models to solve for random effects with known genetic covariances in a computational environment. Students will have developed the computational tools to solve many of the problems they may encounter in their graduate research, while also developing a foundation from which to build new ideas to tackle new problems. 


 % The term project will consist of students finding a genotype-phenotype dataset, and working to develop a genotype to phenotype map and assess genomic predictability throughout the semester, .

% \textbf{Objective:} 
% \begin{itemize}

% 	\item Review of probability
% 	\begin{itemize}
% 		\item Random variables
% 		\item Probability density functions (PDF) and probability mass functions (PMF)
% 		\item Central limit theorem and the normal distribution
% 	\end{itemize} 

% 	\item Single gene model 
% 	\begin{itemize}
% 		\item Additive
% 		\item Dominance 
% 		\item Effect of inbreeding
% 	\end{itemize} 

% 	\item Genetic variances and covariance
% 	\begin{itemize}
% 		\item Additive and dominance covariance
% 		\item Calculating expectations from a pedigree 
% 		\item Calculating realized covariances from marker scores
% 		\item Necessity of marker imputation
% 	\end{itemize} 

% 	\item Linear regression
% 	\begin{itemize}
% 		\item Parameters and likelihood
% 		\item Numerical and categorical predictors 
% 		\item Ordinary least squares
% 		\item Connection to ANOVA/ANCOVA
% 	\end{itemize} 

% 	\item The linear mixed model
% 	\begin{itemize}
% 		\item Random versus fixed effects
% 		\item Known and estimated covariances
% 		\item Expectation Maximization (EM) algorithm for solving mixed model equations 
% 	\end{itemize} 

% 	\item Genome wide association
% 	\begin{itemize}
% 		\item Linkage Disequilibrium
% 		\item Population structure
% 		\item QQ plots
% 	\end{itemize} 

% 	\item Genomic Prediction
% 	\begin{itemize}
% 		\item Evaluation of model fit 
% 		\item Genomic prediction of unobserved individuals
% 		\item k-fold cross-validation 
% 	\end{itemize} 

% \end{itemize}



% \subsection*{Undergraduate degree in computational plant science}

% In the future I seek to help refine the undergraduate degree program in plant science oriented to quantitative genetics and computational biology. The undergraduate degree would have a heavy focus on linear algebra, programming, statistics, machine learning and operational research. 


% Bachelors in unified breeding

% \begin{itemize}

% 	\item Core requirements
% 	\begin{itemize}	
% 		\item Math (linear algebra, matrix calculus)
% 		\item Statistics (prob and Stat theory)
% 		\item Programming (Python / C++ / etc.)
% 		\item Machine learning
% 		\item Operational research (OR dept?)
% 		\item Biology
% 		\item Chemistry
% 	\end{itemize} 

% 	\item PLBR requirements
% 	\begin{itemize}
% 		\item Plant Genetics (PLBRG 2250, Mazourek)
% 		\item Genetic Improvement of Crop Plants (PLBRG 4030, Gore/Reisch)
% 		\item Introduction to quantitative genetics (PLBRG 40XX , Santantonio)
% 		% \item New age of Plant and Animal breeding (GS etc...)
% 	\end{itemize} 

% 	\item Plant Science Electives (must pick two)
% 	\begin{itemize}
% 		\item Pathology 
% 		\item Physiology
% 	\end{itemize} 

% 	\item Management
% 	\begin{itemize}
% 		\item Course about building a successful team
% 	\end{itemize} 

% \end{itemize}

\end{document}
