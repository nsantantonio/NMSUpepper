\documentclass[11pt]{article}
\usepackage{amsmath}
\usepackage{amssymb}
\usepackage[backend=bibtex, style=authoryear]{biblatex}

% \addbibresource{~/Dropbox/TAGreview/PhDreferences.bib}

\textwidth=470pt
\oddsidemargin=0pt
\topmargin=0pt
\headheight=0pt
\textheight=680pt
\headsep=0pt

\title{Statement of Research}


\author{Nicholas Santantonio}
\date{\today}


\newcommand{\gxe}{G$\times$E}
\newcommand{\gxg}{G$\times$G}


\begin{document}

\section*{\centering Statement of Teaching}
\begin{center} Nicholas Santantonio \end{center}

\noindent Plant breeders have traditionally been generalists, combining genetics with a range of plant sciences to identify farmers' needs and produce products to meet those needs. However, the range of skills required in the field is rapidly increasing. Competitive plant breeders are now expected to be proficient in statistics, programming, bioinformatics and computational biology along with the traditional skills in physiology, pathology, agronomy and genetics. As educators, we must decide how to update our curriculum to best prepare students for careers in the post-genomics era of digital agriculture. Should we encourage a mile wide and an inch deep approach? Or should we recognize that the discipline has grown so wide in breadth that we must instead encourage specialization? 

Both will be needed for integrating genomics and digital agriculture into modern breeding programs. Generalists may require instruction in team management, while specialists may need more rigorous courses in areas previously considered outside plant breeding, such as engineering and computer science. Specialists will be vital for collecting and storing large amounts of data that will be used to model and predict crop growth and development. The generalists will coordinate those activities to lead a team that produces new varieties to meet farmers' needs.

Due to the wide breadth of the plant breeding discipline, it may be pertinent to develop multiple paths of instruction. While I believe all future scientists will require some degree of graduate level statistics and computational coursework, a more structured series of quantitative and computational courses would benefit students seeking specialization. Students who chose this path would come out of graduate school with the comprehension and ability to effectively use the latest computational techniques in plant breeding, including the use of genome-wide information for decision making.




\subsection*{Teaching Philosophy}

It is important for students to be exposed to plant breeding ideas from multiple perspectives, and they must given the opportunity to demonstrate critical thinking on different levels. While some students may show analytical thinking and synthesis on exams, others may shine in more hands-on projects. Most mathematical and computational learning occurs through doing, not through watching. The lecture is important to present material in a concise structured manner, but concepts are cemented when the student can reconstruct the ideas on their own time.


Longer term projects provide students the opportunity to practice and apply concepts in a more autonomous environment, and are invaluable for assessing comprehension and critical thinking. Regularly assigned homework and hands-on labs are also crucial for estimation of the pace and overall understanding of the material presented so that adjustments can be made where necessary. 

The greenhouse and field facilities on campus provide good opportunities to get students out of the classroom to see plant breeding in practice. Public databases provide resources where students can find datasets to work with and computer simulations are useful tools to evaluate comprehension, where in order to simulate a system correctly, the student must understand that system well.


All courses I instruct would contain a term project of relevant complexity to augment exams, in-class labs and homework assignments. For quantitatively oriented courses, these projects would consist of a hands-on computational component, where students either find a dataset or use their own data to explore the ideas covered in the course. They would then be asked to present their results in written and oral formats that mirror typical scientific communication. Projects may be team oriented to promote collaborative skills and project management. For courses without a significant quantitative aspect, term projects may be formulated as a research proposal.

I also intend to incorporate more perspectives on how domestication, selection and seed systems work in different cultures through guest lectures and case studies. Review of how seed systems function in other cultures will cultivate discussion on the benefits and consequences of these systems relative to the US and western European systems. 

\subsection*{Courses}

I propose to teach one 400 level undergraduate/graduate course a year, as well as one advanced graduate course every other year to meet the 25\% teaching requirement of the position. The Plant Breeding (AGRO462) course may make a good place to introduce more quantitative ideas to prospective plant breeding students. Additionally, I would like to teach a more advanced quantitative genetics course to complement other quantitative courses, such as AGRO610 and AGRO670.

If I were to instruct AGRO462, the course would include more focus on $21^\text{st}$ century plant breeding concepts. The class would be taught from a historical perspective, allowing the natural progression of plant breeding over the past century to be a teaching tool for how we arrived at the methods we use today. Instead of a textbook, primary literature would be used to introduce key concepts. By the end of the semester, students will be able to think critically about how breeding technology changes through time have impacted the discipline and the societal view thereof. 

I would incorporate more hands-on computational labs and a term project to supplement the literature review framework. Computational labs would be used to augment student understanding of course material, in which students would use available computational tools to analyze example datasets with a focus on interpretation of results. Students would be able to demonstrate critical thinking of plant breeding methods and ideas, with the ability to synthesize when given new plant systems or breeding goals. 


As for the graduate level course, I would like to offer a quantitative genetics theory or statistical genomics course that would include a weekly computational lab and a student term project which would be developed throughout the semester. Starting with basic probability theory and the single locus model, the course would advance through genome-wide association, genomic prediction and selection, as well as selection theory and breeding program optimization. 

Concepts in class would be reinforced through labs and assignments that require students to write their own software to solve plant breeding problems. Assignments would be required to be submitted as typed documents in Markdown, \LaTeX\ or similar format, to expose students to more effective modes of mathematical communication outside of Microsoft Office. The term project might then consist of groups of 2-4 students finding a genotype-phenotype dataset, and working to develop a genotype to phenotype map and assess genomic predictability throughout the semester.




\subsection*{Curriculum}

Currently, most life science students do not acquire in-depth statistics and programming skills until graduate school, impeding their progress while they learn to grapple with these new languages. Moving forward, I would like to work with faculty in PES, statistics, engineering and computer science to build a quantitative/computational genetics undergraduate curriculum. Genetics is a vast field, and it is imperative that students get exposure to and training in the rapidly changing environment in which they will soon be seeking jobs. While ``single-gene'' genetics is still an important field, more and more the community is focused on the complex ``omics'' network that is the foundation of complex organisms. Students in essentially all sub-fields of genetics will need to be able to deal with large datasets, using tailored algorithms to make inferences, predict the unobserved, and guide decision making. Dealing with large data requires skills in programming, linear algebra, statistics and machine learning. 


\medskip

Genomics and digital agriculture is only just starting to change the landscape of food production. Quantitative skills are one of the specializations imperative in plant science, and NMSU must be at the forefront of preparing the best individuals to usher in this new era.

\end{document}
