\pagenumbering{gobble}
\documentclass[11pt, letterpaper]{moderncv}
\usepackage{times, amsmath}

\moderncvstyle{classic}
\moderncvcolor{green}      
\usepackage[utf8]{inputenc}
\usepackage[scale=0.75, margin = 0.75in]{geometry}
\geometry{top  = 20mm}

\name{Nicholas}{Santantonio}
\address{240 Emerson Hall, Ithaca, NY 14853}{(505)~412-2738}

\extrainfo{ns722@cornell.edu}\begin{document}
\recipient{Department of Plant and Environmental Sciences}{College of Agricultural, Consumer and Environmental Sciences\\New Mexico State University\\Skeen Hall Room N127\\  Las Cruces, NM 88003}
\date{\today}
\opening{Dear Hiring Committee,}
\closing{Respectfully yours,}
\makelettertitle


I am delighted to apply to the Chile Pepper Breeding and Genetics Assistant Professor position at the Department of Plant and Environmental Sciences in the College of Agricultural, Consumer and Environmental Sciences at New Mexico State University. I am currently a postdoctoral associate at Cornell University, working with Dr. Kelly Robbins on quantitative genetics solutions for plant breeding. I have combined a strong applied background in small grains and forage breeding programs with theoretical quantitative genetics, allowing me to integrate the newest computational technologies into a successful working breeding program.


I have demonstrated my ability to ask meaningful plant breeding and genetics research questions and communicate the findings of my research through peer-review publications and invited talks at scientific conferences. I have shown an ability to obtain extra-mural funding through a collaborative project to investigate genomic selection strategies in alfalfa. Other collaborations range from projects with colleagues within my institution, to partners in CGIAR programs across the globe, on multiple crop species and genetics/breeding goals. 

I have also demonstrated a strong background in teaching and leadership, serving as a TA for introductory courses in biology and plant breeding during my graduate degree, and currently as a co-instructor for an advanced graduate-level course on genetic modeling in plant breeding. Looking forward, I aim to help prepare future students for data-driven plant breeding by teaching courses with a quantitative emphasis. Computational skills are crucial for leveraging the plethora of data currently being generated in plant science and breeding programs and must be incorporated into plant science and genetics curricula at earlier stages. 


In this data-rich century, a breeding program must adapt to leverage computing capabilities and affordable genotyping technologies. Incorporating genome wide information with remote sensing, I intend to shift from the traditional $20^\text{th}$ century breeding program model to a data-driven $21^\text{st}$ century breeding program to improve trait stability, disease resistance and plant architecture for mechanical harvesting of chile pepper. Implementation of these technologies will not be trivial, and I intend to address several logistical issues by using pepper as a model organism to learn from and demonstrate how to effectively transition to a data-driven breeding program. This transition will provide a valuable resource for public outreach, where farmers and consumers can learn how we are adapting the latest technologies to improving a crop so important to our New Mexican identity, and so close to our hearts. 

As the first graduate of the Genetics degree program at NMSU, I would be honored to have an opportunity to bring my expertise back to my hometown, Las Cruces NM. I would like to thank the hiring committee for considering my application to the Assistant Professor of Chile Pepper Breeding and Genetics. Under the leadership of Drs. Garcia, Nakayama and Bosland, the NMSU Chile Pepper Breeding Program has a rich legacy to which I hope to contribute. 

\vspace{3mm}

Sincerely,\\
\includegraphics[height=1.8cm]{NSsigWhiteBG}\\
Nicholas Santantonio

\end{document}



